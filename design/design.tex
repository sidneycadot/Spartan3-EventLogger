\documentclass[a4paper,twoside]{report}

\usepackage{charter}
\usepackage{parskip}
\usepackage{graphicx}

\title{Spartan-3 Development Board Event Logger}
\author{Sidney~Cadot}

\begin{document}
%
\maketitle
\tableofcontents
%
\chapter{Introduction}

This part is to be written.

\chapter{Design Overview}

See Figure~\ref{fig:dataflow}.

\begin{figure}[h]
\begin{center}
\includegraphics[width=\textwidth]{dataflow}
\end{center}
\caption{Dataflow}
\label{fig:dataflow} % label should come after the caption.
\end{figure}
%
See Figure~\ref{fig:dataflow}.

\chapter{Design Entities}

This part is to be written.

\section{EventLogger}

This is the top-level design element.

Register all inputs to make them clock-synchronous.

Register all outputs to make them clock-synchronous.

\section{InputSection}

Time stamps, sequence numbers, and state vector.

\section{EventFifo}

Wrapper around the GenericFifo.

\section{GenericFifo}

A generic synchronous fifo.

\section{DualPortRAM}

Dual Port RAM, to be implemented using block RAM resources.

\section{EventPrinter}

Pick up an Event from the EventFifo, send a string representation to the SerialTransmitter as a stream of ASCII characters.

At the same time, calculate the CRC-32 of the concatenated bytes that make up the sequence number, timestamp, and bit vectors.

\section{CRC32}

Calculate a standard CRC-32 based on input nibbles.

\section{SerialTransmitter}

Pick up an octet if it is made available, and transmit it to a serial line. Insert start- and stop bits as needed.

\chapter{Future Enhancements}

This part is to be written.

\section{Ethernet backend}

This part is to be written.

\section{Using external RAM resources}

This part is to be written.

\end{document}
